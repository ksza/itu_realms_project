%!TEX root = /Users/mortenq/Documents/ITU/Master/Realms/svn/itu_realms_project/trunk/realms_report/dk.itu.realms.main.tex
\section{Analysis} % (fold)
\label{sec:analysis}
One of our basic assumptions is that features of location based applications can be categorized to create a taxonomy of location features. Having looked at some location-based application from the research literature, we now turn to the smart phone market places. 

\subsection{Popular Location-Based Apps} % (fold)
\label{sub:popular_location_based_apps}
This subsection will describe some popular and interesting location-based apps. We will chose a few that are popular and have a strong reliance on location-based interaction. 

\subsection{foursquare} % (fold)
\label{sub:foursquare}
foursquare is one of the most popular location based application. It's a social platform where users share their locations by \'checking in\'. These check ins generate points and badges, allowing users to compete against each other in a race to get the most points. The competitive gaming aspects is further enhanced by the possibility to become Mayor of a given location - a title awarded the person who checks in the most. 
% subsection foursquare (end)

\subsection{Gowalla} % (fold)
\label{sub:gowalla}
Gowalla was a social location based app that was sold to Facebook in 2012. Similar to foursquare it allowed users to check in at different locations to receive awards. Users were awarded items which could be used in turn to create new places. The app also awarded users \emph{pins} for complete trips; checkins at a series of secified locations.
% subsection gowalla (end)

\subsection{WHERE} % (fold)
\label{sub:where}
WHERE is a location-based app allowing users to share information location specific information. The types of information can be ratings of places, discounts in stores or the price of gas at a gas station. WHERE makes this kind of user-generated information easily available to users in the vicinity. WHERE also has a strong business model where companies can use WHERE to promote themselves. 
% subsection where (end)
% subsection popular_location_based_apps (end)

\subsection{Application Development Systems} % (fold)
\label{sub:application_development_systems}
With explosion of smart phones, businesses have realized the business potential of having an application to promote its products or services. This, in turn, have created a market for app-creation systems, where non-programmers can create simple smart phone applications. Of particular interest of our project is shoutem. shoutem is a web based system for creating smart phone apps. Users can access a webpage to create, edit, and publish iOS and Android applications. shoutem incorporates location-based features by using the foursquare API. Users can add places to the app, and allow check-ins, events (like a party - not programming language events), and user uploaded location-specific comments, media etc. 
% subsection application_development_systems (end)

\subsection{Taxonomy of Location Features} % (fold)
\label{sub:taxanomy_of_location_features}
In this subsection we will try to create a taxonomy for location based features. Using the before described location based apps, we will try to factor out commonalities and abstract them to be able to describe them as general location based features.
% subsection taxanomy_of_location_features (end)
% section analysis (end)
