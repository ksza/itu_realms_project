%!TEX root = /Users/mortenq/Documents/ITU/Master/Realms/svn/itu_realms_project/trunk/realms_report/dk.itu.realms.main.tex
\section{Analysis} % (fold)
\label{sec:analysis}
To gain a better understanding of the type of system we are developing, there are a number of fields we need to explore. Our system will allow users to create Realms - that is digitally enhanced physical locations. The motivation behind our system is the ability to allow end-users to configure these types of interactions. Having described related work in the fields of location-based system and end-user configuration for context-aware computing, we identify two important fields that we will analyze before we proceed; an abstract view on location based systems and the concept of Realms. Furthermore we will need to consider the type of users we design for. At the end of this section, we present a user scenario of our system.

\subsection{Characterization of Location Based Systems} % (fold)
\label{sub:taxanomy_of_location_features}
In this subsection we will try to characterize the properties of location-based systems. To do so we list a number of design dimensions that each describe a property of a location based systems. It is worth noticing that most location-based systems employ several features that does not have anything to do with location. While the foundations of the system may be dependent on location data, a lot of functionality that uses the location data in clever ways are supported. The design dimension we mention here are concerned only with interaction that are location-based. 

\subsubsection{Location Representation}
When we talk about location, we talk about the physical location of a user in the world. It is however important to notice that location can be represented in different ways. In general we can express location in 3 different ways; absolute, symbolic and relative (ref ubicomp book). The absolute location is the location expressed in the absolute measurements as defined by the system. For the earth as our system, the absolute location measurement is a position expressed in terms of latitude and longitude. The symbolic location measurement is a measurement that is meaningful to an interpreter (such as a human). An example would be The university - a location that makes sense for students and employees. Lastly location can be expressed in terms of relative position - that is, a position relative to another- for example 2 kilometers north of Copenhagen (note that Copenhagen is a symbolic location in this example).

\subsubsection{Location Level}
Location type describes the characteristics of the location information used in an interaction. We define two distinct levels to which location systems make use of location information; location-based and location-enhanced. The location-based are interaction in which the outcome or result of the interaction depends on the location of the user. A good example is a search for nearby restaurants using a mobile location-enabled device. The search results from e.g. Google Search, will depend on the location of the user and their sorting likewise. On the other hand, location enhanced interactions are those where location is used to augment an interaction without affecting the outcome of it. An example is a check-in on e.g. Facebook Places. This outcome of the interaction is the check-in of the user - regardless of his location though his location is recorded. Later, a series of location based interactions may occur that follow the location-enhanced one, e.g. if a user want to search for other people in the vicinity using his previously reported location.

\subsubsection{Data Type}
The data type is a dimension we we adopt from Schilit et. al.'s categorization of context aware computing applications \cite{512740}. For data type we distinguish between information and commands. Information is data as information distributed within the system. Commands on the other hand, are executed within the systems. Again, a Goole search for restaurants will return what we characterize as information; a list of restaurants. On the other hand, presenting the user with the ability to print at a nerby printer, or allowing a check-in at a given location is a location-based command.

\subsubsection{Communication Type}
\label{sub:communication.type}
Our last dimension is the communication type. This dimension describes whether a user should query the system for updates - query based -  or whether this happens automatically - push based. An example of the fist one is the, again, a search where a user queries the system for some search results. On the other hand, a system notifying a user when he's close to a given location is an automatic system.  

% subsection taxanomy_of_location_features (end)


\subsection{Design Space} % (fold)
\label{sub:design_space}
To sum up, we have identified 4 dimensions that can be use to characterize a location-based system. These are; location representation, location type, data type, communication type. Based on these dimensions we present our design space for location based systems. Table \ref{tab:designspace} summarizes our design space.
\begin{table}
\begin{tabular}{|l|l|c|r|}
  \hline
  \emph{Loc. representation} & Absolute & Relative & Symbolic \\ \hline
  \emph{Loc. level} & \multicolumn{1}{l}{} & \multicolumn{1}{l|}{Location-Based} & \multicolumn{1}{r|}{Location-Enhanced} \\ \hline
  \emph{Data type} & \multicolumn{1}{l}{} & \multicolumn{1}{l|}{Information} & \multicolumn{1}{r|}{Commands} \\ \hline
  \emph{Loc. level} & \multicolumn{1}{l}{} & \multicolumn{1}{l|}{Query-Based} & \multicolumn{1}{r|}{Push-Based} \\
  \hline
\end{tabular}
	\caption{A design space of location based systems}
  \label{tab:designspace}
\end{table}
% subsection design_space (end) 


\subsection{Users} % (fold)
\label{sub:users}
The motivation behind Realms is to create end-user programming for location-based services. Thus it is important that we design for people without programming experience in mind. The potential users of our system can be people with the need or want to create a location-based service, however without the necessary programming knowledge to build a system themselves. This fact requires us to think carefully about the user-interface of the configurator and abstract all aspects of the system to a level where we expect non-programmers to be able to use it. 
\\\\
We imagine a user-base within marketing industries where a location-based service can be a part of a marketing campaign. In these situations it might not be possible to feasible to create a new app to go with that campaign. We imagine that Realms could be a valuable asset to such campaigns. It would allow companies to easily and quickly create a location based service to go with the campaign. 
% subsection users (end)

\subsection{Expressiveness vs Usability} % (fold)
\label{sub:expressiveness_vs_usability}
Maybe we should have something here about the level of abstraction. I dont know 
% subsection expressiveness_vs_usability (end)


\subsection{Realms} % (fold)
\label{sub:realms}
We introduce the notion of \emph{Realms} to describe a physically confined place augmented with digital information accessible by our systems. This choice is motivated by several factors. The definition of the word \emph{realm} is "A community or territory over which a sovereign rules; a kingdom."\footnote{\url{http://www.thefreedictionary.com/realm}}. This definition has inspired us to choose the word. A realm in our systems is not associated with a kingdom, but it does describe rules and affordances of a territory. The sovereign or kingdom in our system can thus be thought of as the rules of the realms as dictated by the person who configured it. 

Other uses of the word is found in online gaming - such as the online game World of Warcraft\footnote{\url{http://www.worldofwarcraft.com}} - where a realm is a synonym with an instance of the virtual world in which the game resides. This usage also fist well with our thoughts on realms. Indeed when entering a Realm in our system, the user is presented with a new virtual world.
% subsection realms (end)


\subsection{Use Scenario} % (fold)
\label{sub:scenario}
To illustrate the intentions of our proposed system, we present a scenario of its use:
\\\\
\textbf{Configuration Scenario}\\
Tom is the communication responsible for a large danish coffee company with several shops around the Copenhagen area. The company is in the process of developing their new campaign for Coffee-To-Go and wants a location based service as a part of the campaign. The intention of the campaign is to get people to try their new coffee-to-go and the it focusses on the scenario of grabbing a coffee and enjoying it at some of Copenhagen many green areas. Tom uses the Realms system to create a new Realm Coffee-To-Go and marks green areas close to all their shops. The markings, when discovered, contains an advertising text, directions to the closest coffee shop and a discount code to use for a cheap to-go coffee.
\\\\
The campaign goes live with advertisements in local newspapers saying that people can find discount codes for a cup of coffee at green areas in Copenhagen. 
\\\\
\textbf{Mobile Client Scenario}\\
On the way home from word, Maria is reading the local newspaper on the subway. She stumbles across an advertisement from a coffee company that offers cheap to-go coffee for people who have discovered a discount code spread across green areas in Copenhagen through their Realm. Maria decides to get of one stop early and see if she can catch a code, and a cheap coffee in the nice weather. As she exits the subway and launches up the Realms mobile client on her smart phone. The Coffee-To-Go Realm shows up in the list of available Realms and she selects it. She walks to the close by park and explores the Realm by hitting the update button. Indeed, a mark is available that tells that a cheap to-go coffee is available for her of she exits the park at the north exit and finds the coffee shop 
% subsection  (end)
% section analysis (end)


