\section{Discussion} % (fold)
\label{sec:discusion}
In this section we discuss the evaluation results. 
\\\\
The main thing to note before we go in to discussing the evaluation results is, that the general level of programming experience for all our participants was high. As one of the objectives of the system was to bring end-user programming of location based systems to people without programming experience we will not conclude on this matter. We do however believe that the results of the evaluation provide good guidelines for future work on the system. 

\subsection{Usefulness} % (fold)
\label{sub:usefulness}
One of the main objectives of our evaluation was to investigate if users found our system useful. When abstracting the creating of augmented locations to a configuration level, a lot of expressiveness id lost. In our system users could only annotate locations with information or a question. These could then be explored through the mobile phone.
\\\\
All participants in our test however agreed that the system was indeed useful. We received a lot of suggestions for future features, but the basics of just putting information at a location was seen as a first useful way of presenting a system. One of our concerns with our approach was if the limited set of features would create a system that users did not find useful, but the feedback from the test participants leads us to believe that we have taken a right track.
% subsection usefulness (end)

\subsection{Usability} % (fold)
\label{sub:usability}
All participants were positive about the usability of the system. The use of Google maps to help users create Realms and marks was especially well received. The configurator did however suffer from small usability errors that should be improved. One thing we noticed during the test was that users rarely configured the radius of marks. While it seemed obvious to users to play around with the radius of the Realm, they rarely did so with marks. This was also pointed out by a single participant who suggested that marks had no initial radius to force users to think about how big it should be. As the sizes of marks and Realms play a large role in our thoughts of the system, we should consider how to get users to think about the size of marks.
\\\\
The usability of the mobile client was also well received, however as some participants pointed out, the functionality was very limited and so it was hard to not find it usable.
\\\\
All in all the system as a whole was found very ease to use. This leads us to believe that we have taken the right track by using Google maps and a web-interface that relies on text-boxes and buttons.
% subsection usability (end)

\subsection{Features} % (fold)
\label{sub:features}
The most requested feature was some visual feedback on the client regarding the placement of realms and marks. Our idea with not providing any hints as to where to look for marks was to have an exploratory feeling to the system; users should go around the realm and explore it. However almost all participants requested some hints such as a map or textual hints.  
\\\\
The second most requested feature was to allow for more mark types. The two existing types - information and questions - were seen as a good start but to make it more interesting more should be added. Especially mark types that required more interaction between user and system was requested. One user also noted that it would be smart if programmers could write their own marks. This way the simplicity of the system would be preserved - as non programmers could use any existing marks - while experts could write their own. This is indeed a very interesting thought, though it would require a lot of work.
\\\\
Finally two users requested an augmented reality feature. It would indeed to be very interesting if users could use a wide variety of sensors in their phone - like camera and compass - to explore realms.
% subsection features (end)

\subsection{Realms} % (fold)
\label{sub:realms}
When we designed the system we came up with the name of Realms to describe an digitally augmented physical location. We thought of this metaphor as our system should allow user to enter new worlds where other rules might exist. We wanted to investigate how metaphor was perceived by others, so we during our evaluation interviews we asked them about the name and how it related to their experience of the system.
Most of the participants answered that they immediately though of computer games when they heard the word. As one participant put it \emph{"I though it would be a game or something entertaining"}. Only two participants thought the metaphor was good and this leads us to believe that we should either come up with a new metaphor or think about how we can communicate our thoughts behind it. 
% subsection realms (end)
\\\\
All in all the participants of our user study were very positive about our system. The system was fond useful and was easy to use and all participants believed that it was an interesting approach to location-based apps. The most requested features regarded visual feedback on the mobile client to help locate realms and marks. Furthermore we might need to reconsider the name Realms as it was mostly associated with computer games. 

\subsection{Comparison to existing systems} % (fold)
\label{sub:comparison_to_existing_systems}
Here we will look at some of the works on location systems described in the related works and discuss to what extend they could be implemented using Realms instead.
% subsection comparison_to_existing_systems (end)
% section discusion (end)