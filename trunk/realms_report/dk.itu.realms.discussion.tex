\section{Discussion} % (fold)
\label{sec:discusion}
We would like point out before we go in to discussing the evaluation results is, that the general level of programming experience for all our participants was high. As one of the objectives of the system was to bring end-user programming of location based systems to people without programming experience we will not conclude on this matter. We do however believe that the results of the evaluation provide good guidelines for future work on the system. 

\subsection{Usefulness} % (fold)
\label{sub:usefulness}
One of the main objectives of our evaluation was to investigate if users found our system useful. When abstracting the creation of augmented locations to a configuration level, a lot of expressiveness id lost. In our system users could only annotate locations with information or a question. These could then be explored through the mobile phone.
\\\\
All participants in our test however agreed that the system was indeed useful. We received a lot of suggestions for future features, but the basics of putting information at a location was seen as a first useful way of presenting a system. One of our concerns with our approach was if the limited set of features would create a system that users did not find useful, but the feedback from the test participants leads us to believe that we have taken a right track.
% subsection usefulness (end)

\subsection{Usability} % (fold)
\label{sub:usability}
All participants were positive about the usability of the system. The use of Google Maps to help users create Realms and markers was especially well received. The configurator did however suffer from small usability errors that should be improved. One thing we noticed during the test was that users rarely configured the radius of markers. While it seemed obvious to users to play around with the radius of the Realm, they rarely did so with markers. This was also pointed out by a single participant who suggested that markers had no initial radius to force users to think about how big it should be. As the sizes of markers and Realms play a large role in our thoughts of the system, we should consider how to get users to think about the size of markers.
\\\\
The usability of the mobile client was also well received, however as some participants pointed out, the functionality was very limited so it was hard to not find it usable.
\\\\
All in all the system as a whole was found very ease to use. This leads us to believe that we have taken the right track by using Google maps and a web-interface that relies on text-boxes and buttons.
% subsection usability (end)

\subsection{Features} % (fold)
\label{sub:features}
The most requested feature was for the system to be able to push information to the client. Our current implementation has an update button that users can touch to query the server for any markers they are inside. All participants in the study requested this feature which leads us to prioritize this in our future work on the system. Reflecting back to our taxonomy, users wanted to see a push-based communication type. 
\\\\
The second most requested feature was for visual feedback on the client regarding the placement of realms and markers. Our idea with not providing push-based communication or implementing hints as to where to look for markers was to have an exploratory feeling to the system; users should go around the realm and explore it. However almost all participants requested some hints such as a map or textual hints and most participants pointed out, that walking around with the phone in your hand can be quite annoying. 
\\\\
The third most requested feature was to allow for more marker types. The two existing types - information and questions - were seen as a good start but to make the system more interesting more should be added. Especially marker types that required more interaction between user and system was requested. One user also noted that it would be smart if programmers could write their own markers. This way the simplicity of the system would be preserved - as non-programmers could use any existing markers - while experts could write their own. This is indeed a very interesting thought, though it would require a lot of work.
\\\\
Finally two users requested an augmented reality feature. It would indeed to be very interesting if users could use a wide variety of sensors in their phone - like camera and compass - to explore realms.
% subsection features (end)

\subsection{Realms} % (fold)
\label{sub:realms}
When we designed Realms we came up with the name of Realms to describe an digitally augmented physical location. We thought of this metaphor as our system should allow user to enter new worlds where other rules might exist. We wanted to investigate how metaphor was perceived by others, so we during our evaluation interviews we asked them about the name and how it related to their experience of the system.
Most of the participants answered that they immediately though of computer games when they heard the word. As one participant put it \emph{"I though it would be a game or something entertaining"}. Only two participants thought the metaphor was good and this leads us to believe that we should either come up with a new metaphor or think about how we can communicate our thoughts behind it. 
% subsection realms (end)
\\\\
All in all the participants of our user study were very positive about our system. The system was fond useful and was easy to use and all participants believed that it was an interesting approach to location-based apps. The most requested features regarded visual feedback on the mobile client to help locate realms and marks. Furthermore we might need to reconsider the name Realms as it was mostly associated with computer games. 

\subsection{Comparison to existing systems} % (fold)
\label{sub:comparison_to_existing_systems}
We designed and implemented a system that allows users to create simple location-based services through a configuration interface. We now look back at some of the related works we described to discuss to what extend these systems could be created using Realms.
\\\\
The main thing that makes our system unable to create services like Savannah \cite{Benford_Rowland_Hull_Reid_Morrison_Facer_Clayton_2004} and Pirates \cite{Falk:2001:PPI:634067.634140} is the fact that our system only supports information and question markers. While the core of the two system are similar to ours; the physical location of the user enables different functionalities, we simply do not support these functionalities. It would however be possible for us to support more functionalities and thereby getting closer to making these games. It is worth noticing though that the two systems rely on a rich UI to communicate with their users while our UI is very simple and can't be configured.
\\\\
This is also the case when we look at the marketplace applications we described in the related work. foursquare, Gowalla and WHERE all provide very complex functionalities that are not easily abstracted. However, these system we will classify as more complex location-based systems. In the current state of Reals, we consider the configurations to be very simple abstractions of location-based systems. 
\\\\
Looking at Lanibile et. als learning system \cite{4351331} we get a bit closer. We could create a realm where the discovery of a marker would mean that a specific location at a site has been identified or that each location had a question associated with it that would need to be answered to correctly identify the location. The associated information could then guide the users to the next marker. While we cannot configure the realm to give feedback to the users in terms of a 3D representation of a site (or similar) the core basics of discovering a site using the mobile phone to identify a site is indeed possible.
\\\\
All in all it would take a lot more work to come close to be able to support more generic location-based applications. We do however have a system that support the creation of very simple location-based services, and our study suggests that users still find the system useful which encourages us to proceed with the development.
% subsection comparison_to_existing_systems (end)
% section discusion (end)