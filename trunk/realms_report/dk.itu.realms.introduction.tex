%!TEX root = /Users/mortenq/Documents/ITU/Master/Realms/svn/itu_realms_project/trunk/realms_report/dk.itu.realms.main.tex
%I am structuring the introduction according to this guide: http://pages.cpsc.ucalgary.ca/~saul/wiki/pmwiki.php/Chapter1/Deconstruction. It's Saul Greenbarg's guide to thesis chapter 1's, but I think it works fine for smaller projects as well. If we find that the structure doesn't work we can always change it, but for now I think its a good way to do it (as it forces me think about every aspect)

%I've written something down. We probably need to re-write a lot of it till it sounds good
\section{Introduction}
\label{sec.introduction}
With the explosion of location-aware devices in the forms of mobile phones and tablets, location-based systems have become a reality rather than a vision. Programmers have many ways of using the location capabilities of these devices and numerous location aware applications are available for download. A lot of the systems are based on the same principles of basic context-awareness; the selection of available functionality or information, based on the context (location). However, despite the similarities of location-aware programs, developers needs to start from scratch when creating a new one. 
\\\\
In this project we want to explore if it is possible to abstract the creation of simple location-based systems from a programming level to a configuration level.


\subsection{Motivation} % (fold)
\label{sub:context_and_motivation}
The use of location-based systems have risen greatly over the last few years with the introduction of portable locatable devices. Location-based systems demonstrate a successful implementation of context-awareness and the popularity of systems such as foursquare\footnote{\url{http://foursquare.com}} and Facebook places\footnote{\url{http://facebook.com}} show that users have adapted this form of interaction. Companies as well use location information to a great extend, and meanwhile the process of programming location-aware application has become easy for programmers. Operating system and library support for obtaining location data now exists on most mobile platforms, and a few lines of code can yield a precise latitude and longitude of a device. Yet, programming experience is needed even to design the most simple location-based interaction. Holloway and Julien argue that for ubiquitous computing systems to become fully realized in the everyday environment, end-user programming must be available to bridge the gap between system and user \cite{Holloway:2010:CEP:1882362.1882398}. This will also serve as our motivation; to empower end users with the ability to augment physical locations with digital information in a simple non-programming way.
% subsection context_and_motivation (end)'

%We mention related work in a section so we leave it out of the introduction

\subsection{Hypothesis} % (fold)
\label{sub:hypothesis}
It is our hypothesis that some common features of location-based systems can be identified, and that these features can be supported by a system designed to create location-based interaction. We want to develop a software infrastructure that, when configured, augments a certain physical area with digital capabilities. To elaborate, we introduce the notion of 'Realms'. A Realm is a physically confined place augmented with digital information. A Realm is always associated with a physical place, however different Realms can provide different functionalities. Access to the realms is provided through a mobile phone application, so only one app is needed to access any number of realms with any number of affordances. Managing of realms is handled by a server that phones connect to and the creation of Realms is handled through a web-interface. 
% subsection hypothesis (end)

\subsection{Goals and Methods} % (fold)
\label{sub:goals_and_methods}
Our goals with this project are three-fold:
\begin{enumerate}
	\item We will investigate existing location based apps and systems and identify a taxonomy of location-based systems.
	\item We will implement a Realms system that allows for the creation of augmented physical spaces through configuration. 
	\item We will present an evaluation of the system and discuss its possibilities and limitations.
\end{enumerate}

\noindent To reach these goals, we will take the following steps:

\begin{enumerate}
	\item We will study a number of popular location-based applications and group their functionalities. We will then generalize these functionalities to identify the taxonomy.
	\item We will build the Realms system; a two part system consisting of a web-interface - the configurator - and a mobile client.
	\item We will evaluate our system in two ways; (1) we will have users try the configurator and mobile client to study the effectiveness and ease of use of our system, and (2) we will compare the functionalities of our system to those found in other systems to conclude on the possibilities and limitations of our system.
\end{enumerate}
% subsection goals_and_methods (end)

\subsection{Contributions} % (fold)
\label{sub:contributions}
Our main contributions follow our goals:
\begin{itemize}
	\item We present a taxonomy for location-based application functionalities. 
	\item We present the Realms system. A software platform for augmenting physical locations through configuration.
	\item We present an discussion of the feasibility of our implementation based on a user evaluation.
\end{itemize}
% subsection contributions (end)

\subsection{Overview} % (fold)
\label{sub:overview}
The remainder of the report is structured as follows. Section \ref{sec:related_work} describes related work on location-based systems and end user programming. Section \ref{sec:analysis} presents a taxonomy of location application functionalities and present our thought on the metaphor Realm. Section \ref{sec:design} present our design rationale for the Realms system and  Section \ref{sec.implementation} describes the implementation details. Section \ref{sec.eval} presents our evaluation of the system and followed by a discussion of the results in Section \ref{sec:discusion}. Section \ref{sec.conclusions} concludes the current work and suggests directions for future work in the area.
% subsection overview (end)
