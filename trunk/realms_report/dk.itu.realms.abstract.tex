\noindent With the rise in numbers of smart phones equipped with location-sensing technologies, location-based applications have become increasingly popular and are found in large numbers around the mobile phone application marketplaces. Location-based applications use the physical location of the user to allow for new ways of interacting with software or with other people.
\\\\
To further push location-based systems in to every day use, we present an end-user programming system - Realms - that allows users to augment physical locations with digital information. With Realms, users can augment physical locations with information through a web-interface centered around a simple map. The information is thereafter made available on the mobile phone application depending on basic rules and the behavior of the mobile phone user
\\\\
We describe our motivations, design rationale and implementation of the system and present a preliminary evaluation based on a test with 9 users. This evaluation showed that the participants found the system useful and easy to use and pointed to work on push-based communication and a more visual mobile client.